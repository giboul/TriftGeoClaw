\documentclass[11pt]{article}
\usepackage{packages}
\DeclareSIUnit{\msm}{m.s.m.}

\title{Numerical Modeling of Impulse Waves Generated by Avalanches}
\author{Axel Giboulot\\Supervised by Christophe Ancey}
\date{\today}

\begin{document}

\maketitle

\tableofcontents

\clearpage

% Opening

\section*{Abstract}

Impulse waves\footnote{An impulse wave is a wave created when a flow (avalanche, landslide, or other) plunges into a body of water (such as a lake, sea, or ocean).}
have caused past disasters with thousands of casualties \citep{Vajont, Palcacocha}.
Whereas dense flows such as landslides and icefall collapses have been responsible for such events,
the drainage of Trützisee \citep{Trutzisee} by an avalanche suggests that avalanches may be just as capable.

The only method available today to estimate the height of an impulse wave is an empirical relation, derived from dimensional analysis \citep{heller}.
This relation has led in practice to unrealistic predictions, and laboratory experiments have shown that it underestimates density effects \citep{zitti}. % and that flow cohesion (absent from the empirical relation) also matters \citep{zhenzu}.
Given these limitations, an alternative method is needed to estimate the height of impulse waves, particularly for avalanches.

The solution proposed here is to assume a sudden transformation of snow into water.
Thus, it suffices to impose the avalanche in the tsunami model after correcting the fluxes (mass and momentum) by the density of snow.
This enables the separate use of existing numerical avalanche and tsunami models.

This method has the advantage of being physically based (Saint-Venant equations) and conservative, since the energy dissipation associated with shock absorption due to snow compressibility is neglected.
The wave energy is therefore limited to that of the avalanche, and the resulting waves are all the more realistic.

The case of the future Trift dam \citep{manso} is considered here, as the reservoir’s exposure to avalanches makes it an ideal case study.

% Background
\section{Introduction}

An avalanche generates an \emph{impulse wave} when it strikes a body of water such as a lake.
This wave may damage areas near or downstream of the lake depending on its amplitude.
While known historical events were triggered by dense flows such as landslides \citep{Vajont} and icefalls \citep{Palcacocha},
the question for the Trift dam is whether avalanches can also cause large waves.
More generally, we ask: how can one estimate impulse waves generated by avalanches?

\section{Approach}

The analytical approach to the problem is currently limited to attempts at nondimensionalization \citep{heller}.

This option is rejected here because the effect of snow's low density is poorly captured \citep{zitti}.
The absence of physical limits and the ambiguity of defining the depth $h_\ast$ in practical cases further motivate rejecting this approach.
Numerical modeling therefore becomes advantageous and much cheaper than a physical model.

The drawback is that a single model cannot suffice, since the rheology of snow differs fundamentally from that of water.
Using one model dedicated to avalanches and another dedicated to waves simplifies the equations to solve and allows using existing numerical models.
To link the avalanche model to the wave model,
we choose to assume an instantaneous transformation of snow into water at a certain distance from the reservoir.

Such a sudden transformation must not violate the governing equations.
In practice, the Saint-Venant equations are used for both avalanches and waves.
They express mass and momentum conservation.
Thus, the transformation must preserve these two quantities.

\section{Instantaneous Transformation of Snow into Water:\\An Equivalent Avalanche}

The principle of Saint-Venant models is to vertically average the local mass and momentum conservation equations for an incompressible fluid and remove weakly influential terms.
For each surface element $\mathrm dA$, the unknowns are the mass $\varrho h~\mathrm dA$ and momentum $\varrho h\bm u~\mathrm dA$ of the column.
Define two avalanches $\mathcal A$ and $\mathcal A'$ as equivalent if their mass $m$ and $m'$ and momentum $\bm q$ and $\bm q'$ are equal everywhere.
For two such avalanches with different densities $\varrho$ and $\varrho'$,
their flow depths are related by the factor $\kappa\doteq\varrho/\varrho'$.
When $\varrho'=\varrho_\text{water}$, $\kappa$ is simply the inverse relative density of avalanche $\mathcal A$.

$$\begin{pmatrix}m\\\bm q\end{pmatrix} =\begin{pmatrix}m'\\\bm q'\end{pmatrix}
\Rightarrow\begin{pmatrix}\varrho h\\\varrho h \bm u\end{pmatrix} = \begin{pmatrix}\varrho' h'\\\varrho' h' \bm u'\end{pmatrix}$$

\begin{equation}\label{eq:equiv}
    \Rightarrow \begin{pmatrix}h\varrho/\varrho' \\\bm u\end{pmatrix} = \begin{pmatrix}h'\\\bm u'\end{pmatrix}
\end{equation}

Thus, for avalanche $\mathcal A'$ of density $\varrho'$ to be equivalent to avalanche $\mathcal A$ of density $\varrho$, the flow depth must be corrected by $\kappa=\varrho/\varrho'$.

The assumption of instantaneous transformation is energy-conservative: shock absorption due to snow compressibility is neglected.
This assumption corresponds to an elastic impact with a restitution coefficient of \SI{100}{\percent}.

\begin{figure}
    \centering
    \includesvg[width=0.8\linewidth]{media/avalanche_equivalente.svg}
    \caption{Two avalanches of different densities but equal mass $\left(m=\iint\varrho h~\mathrm dA=\iiint2\varrho h/2~\mathrm dA\right)$ and momentum $\left(q=\iint\varrho hu~\mathrm dA=\iint2\varrho h/2 u~\mathrm dA\right)$.}
    \label{fig:avalanches_equivalentes}
\end{figure}

\section{Boundary Condition and Information Conflict}

Once the equivalent avalanche is defined, it must be inserted correctly through boundary conditions.
Typically, coupling two numerical models requires information exchange in both directions.
However, flowing avalanches are usually supercritical.
This means that with a \emph{properly oriented} boundary condition, the wave model does not need to send information back to the snow model because information cannot propagate upstream.
This is highly advantageous: one can simply extract results from the avalanche model and input them into the wave model without explicit coupling.

To clarify what a \emph{properly oriented} boundary condition is, we examine the influence region of a perturbation.
For first-order advection–diffusion, a perturbation propagates over $\delta t$ by a translation $\bm u ~\delta t$ plus diffusion $\sqrt{gh} ~\delta t$.
When flow is subcritical ($u<\sqrt{gh}$), the perturbation spreads in all directions.
When supercritical ($u>\sqrt{gh}$), the perturbed region forms a cone whose angle depends on the Froude angle $\varphi\doteq\arcsin(1/\mathrm{Fr})$, as shown in \autoref{fig:cone_propagation}.
A conflict occurs when this perturbed region intersects the boundary.
One way to avoid this risk is to impose a boundary condition that follows a contour line.
Avalanches tend to follow the slope, flowing perpendicular to contour lines and therefore to the boundary, avoiding information conflicts provided the flow remains supercritical.

\begin{figure}
    \centering
    \includesvg[width=0.7\linewidth]{media/froude_en.svg}
    \caption{Propagation cone for information in a linear advection, supercritical regime. Circles show the extent of a perturbation at time $t_0+~\delta t$. The flow is supercritical while $\varphi=\arcsin\left(\sqrt{gh}/u\right)<\SI{90}{\degree}$, which implies $\mathrm{Fr}>1$.
    Information conflict occurs when $\beta_1=\pi-\alpha-\varphi<0$ or $\beta_2=\alpha-\varphi<0$.}
    \label{fig:cone_propagation}
\end{figure}

\section{Practical Implementation:\\Insertion of the Equivalent Avalanche}

For impulse waves caused by avalanches, the reference avalanche consists of snow (density $\SI{300}{\kilo\gram\per\meter\cubed}<\varrho_\text{snow}<\SI{500}{\kilo\gram\per\meter\cubed}$) and the equivalent avalanche consists of water ($\varrho_\text{water}=\SI{1000}{\kilo\gram\per\meter\cubed}$).
Thus, the flow depth $h_\text{snow}$ must be corrected by $\kappa=\varrho_\text{water}/\varrho_\text{snow}$ to obtain $h_\text{water}=\kappa h_\text{snow}$ according to \autoref{eq:equiv}.

Because the rheology of snow and water differs, the distance traveled by the wave while still represented as water must be minimized so that the mass and momentum fluxes at impact match those of the snow model.
In other words, the boundary elevation $z_b$ must be slightly higher than the lake level $z_l$ by an offset $\delta z=z_b-z_l$.
This offset must be sufficient to allow possible interaction between avalanche and wave.

To avoid limiting the runup\footnote{Runup is the upward surge of water on shorelines.} at $z_b$, the zones not reached by the avalanche must remain free.
The boundary therefore follows the edges of the computational domain, detouring around the avalanche without descending below $z_b$. A summary diagram is shown in \autoref{fig:resume}.

\begin{figure}
    \centering
    \includesvg[width=\linewidth]{media/simulations_en.svg}
    \caption{Simulation schematics.
    \textit{(a)} The avalanche is first simulated with the snow model, then its quantities $(h, hu, hv)^\intercal$ are scaled by $\kappa=\varrho_\text{water}/\varrho_\text{snow}$.
    \textit{(b)} The equivalent avalanche is then introduced into the water model above a contour line slightly higher than the lake.
    This line must be close to the lake level to minimize the distance the equivalent avalanche travels as water.
    \textit{(c)} The equivalent avalanche may also be introduced through a curvilinear boundary condition (following the contour line) to lighten computations, at the cost of reducing space available for runup.
    An implementation of \textit{(b)} with \textsc{GeoClaw} is available online: \href{https://github.com/giboul/TriftGeoClaw}{https://github.com/giboul/TriftGeoClaw}.}
    \label{fig:resume}
\end{figure}

\section{Case Study at Trift}

After more than \SI{2}{\kilo\meter} of retreat of the Trift Glacier,
a dam is planned directly downstream of the glacier \citep{manso}.
Located at \SI{1767}{\msm} and surrounded by steep slopes, the reservoir will be exposed to avalanches,
making it an ideal case study to test the proposed method.
An example result is shown in \autoref{fig:Trift}, where avalanches are first simulated with AVAC\footnote{
    AVAC is a GeoClaw-based code for simulating dense avalanches, available online:
    \href{https://github.com/cancey/AVAC}{https://github.com/cancey/AVAC}.
}.
The impulse wave reaches a maximum height of \SI{5}{\meter} and disperses before reaching the dam.

Overflow is caused by a combination of:
\begin{itemize}
    \item upon impacting the dam, the wave converts its kinetic energy into potential energy, greatly increasing its height, and
    \item wave reflections on the shores occasionally combine to form a transient, larger wave.
\end{itemize}

This second phenomenon cannot be captured by a simple analytical method because it depends directly on the two-dimensional topography.

\begin{figure}
    \centering
    \includesvg[pretex=\scriptsize,width=0.33\linewidth]{media/snapshot_1.svg}%
    \includesvg[pretex=\scriptsize,width=0.33\linewidth]{media/snapshot_2.svg}%
    \includesvg[pretex=\scriptsize,width=0.33\linewidth]{media/snapshot_3.svg}
    \caption{Simulation results of an impulse wave at Trift showing the free-surface variation $s=h-h_0$.
    Left: avalanche at initial time ($t=0$),
    middle: impulse formation ($t=\SI{17}{\second}$),
    right: final state ($t=\SI{1}{\minute}~\SI{20}{\second}$).}
    \label{fig:Trift}
\end{figure}

The solution is visually satisfactory.
Physically, experiments show that the peak mechanical energy of the impulse wave is about \SI{15}{\percent} of that of the avalanche \citep{zitti,zhenzu}.

This relation indicates that the results obtained are realistic, whereas the empirical formula predicts a \SI{20}{\meter} wave.
\autoref{fig:energie} compares the regressions of \cite{zitti} and \cite{zhenzu} with the energies obtained.

\begin{figure}
    \centering
    \resizebox{0.8\textwidth}{!}{\input{media/energies_en.pgf}}
    \caption{Ratios between mechanical energy of the wave and that of the avalanche.}
    \label{fig:energie}
\end{figure}

\section{Limitations of the Method}

Here, we introduce only the mass and momentum fluxes with a density correction.
The effects of more subtle quantities are not represented.
For example, flow cohesion—important in wave formation—is absent \citep{zhenzu}.
The same applies to snow compressibility.
The complexity of the snow–water mixture is also ignored because solving two-phase Saint-Venant equations is computationally expensive\footnote{
The D-Claw project \citep{dclaw} solves two-phase Saint-Venant equations.
}.

\section{Conclusion}

Although the interaction between two phases around an impulse wave posed difficulties,
the assumptions of instantaneous snow–water transformation and supercritical flow allowed us to approximate and implement the problem\footnote{\href{https://github.com/giboul/TriftGeoClaw}{https://github.com/giboul/TriftGeoClaw}}.
The proposed method makes it possible to simulate an impulse wave given a topographic survey and a prior avalanche simulation.
Even if phase interaction details are neglected, the method offers a credible alternative to empirical relations.

\clearpage

\bibliography{refs}

\end{document}
