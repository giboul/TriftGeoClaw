% Opening
% \section*{Abstract - EN}

% The Shallow Water Equations are used by practitioners to simulate both waves and avalanches. Since they assume a constant unit weight, they cannot be used as such to model impulse waves induced by snow avalanches diving into a lake. To keep using the existing numerical models to impulse waves, we suggest implementing an equivalent water avalanche whose mass and momentum are that of the snow avalanche.

\section*{Résumé}

Les vagues d'impulsion\footnote{Une vague d'impulsion est une vague créée suite à la pénétration d'un écoulement (avalanche, glissement de terrain ou autre) dans un corps d'eau (tel qu'un lac, une mer ou un océan).}
ont provoqué par le passé des catastrophes dont les victimes se comptent par milliers \citep{Vajont, Palcacocha}.
Alors que des écoulements lourds sont à l'origine de ces événements (glissement de terrain et chute de sérac),
la vidange du Trützisee \citep{Trutzisee} par une avalanche laisse penser que les avalanches en sont tout aussi capables.

La seule méthode disponible à ce jour pour estimer la hauteur d'une vague d'impulsion est une relation empirique, tirée d'une analyse dimensionnelle \citep{heller}.
Cette relation a conduit à des prédictions irréalistes dans la pratique et des expériences en laboratoire ont montré qu'elle sous-estime les effets de la densité \citep{zitti}. % et que la cohésion de l'écoulement (absente dans la relation empirique) a son importance \citep{zhenzu}.
Face à ces limites, il est nécessaire de trouver une alternative pour estimer la hauteur de la vague d'impulsion, en particulier pour les avalanches.

La proposition faite ici est de supposer une transformation subite de la neige en eau.
Ainsi, il suffit d'imposer l'avalanche dans le modèle de tsunami après correction des flux (de masse et de quantité de mouvement) par la densité de la neige.
Cela permet d'employer les modèles numériques d'avalanches et de tsunamis existants, séparément.

Cette méthode a l'avantage d'avoir une base physique (les équations de Saint-Venant) et d'être conservatrice car la dissipation énergétique liée à l'absorption du choc par la compressibilité de la neige est négligée.
L'énergie de la vague se limite en conséquence à celle de l'avalanche et les vagues résultantes en sont d'autant plus réalistes.

Le cas du futur barrage de Trift \citep{manso} est traité ici car l'exposition de sa retenue aux avalanches en fait un parfait cas d'étude.

% Background
\section{Introduction}

Une avalanche provoque une \emph{vague d'impulsion} lorsqu'elle percute un corps d'eau comme un lac.
Cette vague peut éventuellement endommager le voisinage ou l'aval du lac selon son amplitude.
Alors que les événements connus furent provoqués par des écoulements lourds comme des glissements de terrain \citep{Vajont} et des chutes de sérac \citep{Palcacocha},
la question qui se pose pour le barrage Trift est de savoir si les avalanches peuvent également causer des vagues de grande ampleur.
Cette question est ici généralisée : comment peut-on approcher la vague d'impulsion dans le cas des avalanches ?
